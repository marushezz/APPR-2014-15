\documentclass[a4paper, 11pt]{article}
\usepackage[utf8]{inputenc}
\usepackage{lmodern}
\usepackage[T1]{fontenc}
\usepackage[slovene]{babel}
\usepackage{eurosym}
\usepackage{graphicx}
\usepackage{url}


\pagestyle{plain}


\begin{document}

\title{Poročilo pri predmetu \\
Analiza podatkov s programom R}
\author{Maruša Valant}
\maketitle{Analiza regionalne bruto dodane vrednosti v osnovnih cenah po letu,standardni klasifikaciji dejavnosti in regijah} 

\section{O projektu}

V projektu bom analizirala podatke o bruto dodani vrednosti po dejavnostih v Sloveniji v letih 2000-2013. Podatke sem pridobila iz Statističnega urada Republike Slovenije:

\begin{itemize}
\item{\url{http://pxweb.stat.si/pxweb/Dialog/varval.asp?ma=0309202S&ti=Regionalna+bruto+dodana+vrednost+po+dejavnostih+v+osnovnih+cenah%2C+teko%E8e+cene%2C+Slovenija%2C+letno&path=../Database/Ekonomsko/03_nacionalni_racuni/30_03092_regionalni_rac/&lang=2}}
\item{\url{http://www.stat.si/novica_prikazi.aspx?id=6254}}

\end{itemize}

\section{Obdelava, uvoz in čiščenje podatkov}

Uvozila sem podatke v obliki csv in spletne strani

\begin{itemize}
 \item{\url{http://pxweb.stat.si/pxweb/Dialog/varval.asp?ma=0309202S&ti=Regionalna+bruto+dodana+vrednost+po+dejavnostih+v+osnovnih+cenah%2C+teko%E8e+cene%2C+Slovenija%2C+letno&path=../Database/Ekonomsko/03_nacionalni_racuni/30_03092_regionalni_rac/&lang= }} 
\end{itemize}
in podatke v obliki html iz spletne strani 
\begin{itemize}
\item{\url{http://www.stat.si/novica_prikazi.aspx?id=6254n}}.
\end{itemize}

Po uvozu podatkov sem oblikovala tortni graf, ki predstavlja deleže dodanih vrednosti v Sloveniji po dejavnostih v letu 2012.
\newpage
\textbf{Legenda za tortni graf:}
\begin{table}[h]
\begin{tabular}{lll}
A & Kmetijstvo, gozdarstvo in ribištvo \\
BCDE & Predelovalne dejavnosti, rudarstvo in druga industrija \\
C & Predelovalne dejavnosti \\
F & Gradbeništvo \\
GHI & Trgovina, gostinstvo, promet \\
J & Informacijske in komunikacijske dejavnosti\\
K & Finančne in zavarovalniške dejavnosti\\
L & Poslovanje z nepremičninami \\
MN & Strokovne, znanstvene, tehnične in druge posl. dejavnosti \\
OPQ & Uprava in obramba, obv. soc. varnost, izob., zdravstvo \\
RSTU & Druge dejavnosti
\end{tabular}
\end{table}

\begin{center}
\includegraphics[width=\textwidth]{../slike/graf.pdf}
\end{center}






Za primerjavo bruto dodane vrednosti v Sloveniji v letih 2012 in 2013 sem naredila še stolpični graf tako da je za vsako dejavnost skupaj dodana vrednost v letu 2012 (v  modri barvi) in zraven dodana vrednost v letu 2013 (v rdeči barvi). 
\newline
\textbf{Legenda kratic na x-osi v stolpičnem grafu:}
\begin{table}[h]
\begin{tabular}{lll}
A  & Kmetijstvo in lov,gozdarstvo,ribištvo \\
B & Rudarstvo  \\
C & Predelovalne dejavnosti	\\
D & Oskrba z el.energijo,plinom in paro	\\
E & Oskr.z vodo;rav.z odpl.,odp.;san.okolja	\\
F & Gradbeništvo	\\
G & Trgovina;vzdrž.in popravila mot.vozil	\\
H & Promet in skladiščenje	\\
I & Gostinstvo	\\
J & Informacijske in komunikacijske dej.\\	
K & Finančne in zavarovalniške dej.	\\
L & Poslovanje z nepremičninami	\\
M & Strokovne,znanstvene in tehnične dej.	\\
N & Druge raznovrstne poslovne dej.	\\
O & Javna uprava in obramba;obv.soc.varnost \\
P & Izobraževanje	\\
Q & Zdravstvo in socialno varstvo	\\
R & Kulturne,razvedrilne in rekreac.dej.\\	
S & Druge dejavnosti
\end{tabular}
\end{table}


\begin{center}

\includegraphics[width=\textwidth]{../slike/plot.pdf}

\end{center}


\section{Analiza in vizualizacija podatkov}



\section{Napredna analiza podatkov}



\end{document}