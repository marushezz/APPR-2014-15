\documentclass[11pt,a4paper]{article}

\usepackage[slovene]{babel}
\usepackage[utf8x]{inputenc}
\usepackage{graphicx}

\pagestyle{plain}

\begin{document}
\title{Poročilo pri predmetu \\
Analiza podatkov s programom R}
\author{Maruša Valant}
\maketitle

\section{Analiza regionalne bruto dodane vrednosti v osnovnih cenah po letu,standardni klasifikaciji dejavnosti in regijah} 

V projektu bom analizirala podatke o bruto dodani vrednosti po dejavnostih v Sloveniji v letih 2000-2013. Podatke sem pridobila iz Statističnega urada Republike Slovenije:
-http://pxweb.stat.si/pxweb/Dialog/varval.asp?ma=0309202S&ti=Regionalna+bruto+dodana+vrednost+po+dejavnostih+v+osnovnih+cenah%2C+teko%E8e+cene%2C+Slovenija%2C+letno&path=../Database/Ekonomsko/03_nacionalni_racuni/30_03092_regionalni_rac/&lang=  
-http://www.stat.si/novica_prikazi.aspx?id=6254

\section{Obdelava, uvoz in čiščenje podatkov}

Uvozila sem podatke v obliki csv in spletne strani http://pxweb.stat.si/pxweb/Dialog/varval.asp?ma=0309202S&ti=Regionalna+bruto+dodana+vrednost+po+dejavnostih+v+osnovnih+cenah%2C+teko%E8e+cene%2C+Slovenija%2C+letno&path=../Database/Ekonomsko/03_nacionalni_racuni/30_03092_regionalni_rac/&lang= 
in podatke v obliki html iz spletne strani http://www.stat.si/novica_prikazi.aspx?id=6254.

Po uvozu podatkov sem oblikovala tortni graf, ki predstavlja deleže dodanih vrednosti v Sloveniji po dejavnostih v letu 2012.

\includegraphics{../slike/graf.pdf}



\section{Analiza in vizualizacija podatkov}

\includegraphics{../slike/povprecna_druzina.pdf}

\section{Napredna analiza podatkov}

\includegraphics{../slike/naselja.pdf}

\end{document}
