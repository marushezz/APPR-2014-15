\documentclass[a4paper, 11pt]{article}
\usepackage[utf8]{inputenc}
\usepackage{lmodern}
\usepackage[T1]{fontenc}
\usepackage[slovene]{babel}
\usepackage{eurosym}
\usepackage{graphicx}
\usepackage{url}



\pagestyle{plain}


\begin{document}



\author{Maruša Valant}
\title{Analiza regionalne bruto 
dodane vrednosti v osnovnih cenah 
po letu,standardni klasifikaciji 
dejavnosti in regijah} 
\maketitle

\section{O projektu}
V projektu bom analizirala podatke o bruto dodani vrednosti po dejavnostih v Sloveniji v letih 2000-2013. Podatke sem pridobila iz Statističnega urada Republike Slovenije:

\begin{itemize}

\item{\url{http://www.stat.si/novica_prikazi.aspx?id=6254}}
\item{\url{http://pxweb.stat.si/pxweb/Database/Regije/Regije.aspn}}

\end{itemize}

\section{Obdelava, uvoz in čiščenje podatkov}

Uvozila sem podatke o bruto dodani vrednosti v Sloveniji v letih od 2000 do 2012 v obliki csv(ime tabele je DODANAVRED) iz spletne strani Statističnega urada Republike Slovenije. Tabela vsebuje:

\begin{enumerate}
\item{posamezne regije(imenska spremenljivka)}
\item{dejavnosti(imenska spremenljivka)}
\item{strukturo po regijah in dejavnostih(urejenostni spremenljivki)}
\item{dodane vrednosti po regijah in dejavnostih(številski spremenljivki)}

\end{enumerate}

Nato sem uvozila še tabelo(dodanavrednost) v obliki html, ki vsebuje podatke za leti 2012 in 2013, iz spletne strani: 
\begin{itemize}
\item{\url{http://www.stat.si/novica_prikazi.aspx?id=6254n}}.
\end{itemize}
V tabeli so podane samo številske spremenljivke o bruto dodani vrednosti za celotno Slovenijo in posamezne regije.

Po uvozu podatkov sem oblikovala tortni graf, ki predstavlja deleže dodanih vrednosti v Sloveniji po dejavnostih v letu 2012.
\newpage
\textbf{Legenda za tortni graf:}
\begin{table}[h]
\begin{tabular}{lll}
A & Kmetijstvo, gozdarstvo in ribištvo \\
BCDE & Predelovalne dejavnosti, rudarstvo in druga industrija \\
C & Predelovalne dejavnosti \\
F & Gradbeništvo \\
GHI & Trgovina, gostinstvo, promet \\
J & Informacijske in komunikacijske dejavnosti\\
K & Finančne in zavarovalniške dejavnosti\\
L & Poslovanje z nepremičninami \\
MN & Strokovne, znanstvene, tehnične in druge posl. dejavnosti \\
OPQ & Uprava in obramba, obv. soc. varnost, izob., zdravstvo \\
RSTU & Druge dejavnosti
\end{tabular}
\end{table}

\begin{center}
\includegraphics[width=\textwidth]{../slike/graf.pdf}
\end{center}


Iz grafa je razvidno, da so največji dele\-ž bruto dodane vrednosti v Sloveniji leta 2012 prinesle \textit{predelovalne dejavnosti, rudarstvo in
druga industrija}, sledijo ji \textit{trgovina, gostinstvo, promet}, najmanj pa so prispevale \textit{kmetijstvo, gozdarstvo in ribi\-štvo}. \\
Za primerjavo bruto dodane vrednosti v Sloveniji v letih 2012 in 2013 sem naredila še stolpični graf tako da je za vsako dejavnost skupaj dodana vrednost v letu 2012 (v  modri barvi) in zraven dodana vrednost v letu 2013 (v rdeči barvi).\\



\textbf{Legenda kratic na x-osi v stolpičnem grafu:}
\begin{table}[h]
\begin{tabular}{lll}
A  & Kmetijstvo in lov,gozdarstvo,ribištvo \\
B & Rudarstvo  \\
C & Predelovalne dejavnosti  \\
D & Oskrba z el.energijo,plinom in paro  \\
E & Oskr.z vodo;rav.z odpl.,odp.;san.okolja	\\
F & Gradbeništvo	\\
G & Trgovina;vzdrž.in popravila mot.vozil	\\
H & Promet in skladiščenje	\\
I & Gostinstvo	\\
J & Informacijske in komunikacijske dej.\\	
K & Finančne in zavarovalniške dej.	\\
L & Poslovanje z nepremičninami	\\
M & Strokovne,znanstvene in tehnične dej.	\\
N & Druge raznovrstne poslovne dej.	\\
O & Javna uprava in obramba;obv.soc.varnost \\
P & Izobraževanje	\\
Q & Zdravstvo in socialno varstvo	\\
R & Kulturne,razvedrilne in rekreac.dej.\\	
S & Druge dejavnosti
\end{tabular}
\end{table}

Iz grafa lahko vidimo, da v letu 2013 napram 2012 ni večjih sprememb.Malo je zrasla dodana vrednost v predelovalni dejavnosti, oskrbi z el.energijo,plinom in paro, oskrbi z vodo, prometu in skladiščenju, padla pa v rudarstvu, gradbeništvu, finančnih in zavarovalniških dejavnostih, ter poslovanju z nepremičninami.\\


\begin{center}

\includegraphics[width=\textwidth]{../slike/plot.pdf}
\end{center}

\section{Analiza in vizualizacija podatkov}
Za boljšo predstavitev podatkov sem uvozila zemljevid Slovenije. Na njem so označene posamezne regije, glavno mesto Slovenije - Ljubljana, je označeno z rdečim kvadratom, različne barve regij pa nam povejo kako produktivne so bile posamezne regije leta 2012.
\newpage
\begin{center}
\includegraphics[width=\textwidth]{../slike/Slov.pdf}
\end{center}

Iz zemljevida je razvidno, da je leta 2012 največji del bruto dodane vrednosti v Sloveniji prispevala Osrednjeslovenska (36.8\%), najmanj pa Zasavska (1.4\%).


\section{Napredna analiza podatkov}

Za analizo podatkov sem uvozila še tabelo (prebivalstvo) v obliki csv iz spletne strani:
\begin{itemize}
\item{\url{http://pxweb.stat.si/pxweb/Dialog/varval.asp?ma=05A2010S&ti=&path=../Database/Dem_soc/05_prebivalstvo/05_osnovni_podatki_preb/10_05A20_prebivalstvo_letno/&lang=2}}
\end{itemize}
v kateri so podatki o številu prebivalcev Slovenije v letih od 1960 do 2013.
Na podlagi podatkov od let 2000 do 2012 iz te tabele(prebivalstvo) in podatkov o bruto dodani vrednosti po posameznih letih v Sloveniji iz tabele DODANAVRED, sem narisala graf s pomočjo funkcije gam. \\
Opazimo, da v letih od 2000 do 2008, rasteta tako število prebivalcev kot BDP, od leta 2008 naprej, pa število prebivalcev raste BDP pa pada. Razlog za padec BDP je najverjetneje gospodarska kriza, ki se je začela ravno leta 2008.

\includegraphics[width=\textwidth]{../slike/prebBDP.pdf}

Zaradi padca BDP, ne moremo sklepati, da sta rast prebivalstva in BDP povezana. Iz tega razloga sem naredila napovedi ločeno.

S pomočjo funkcije ts, ki je ustvarila časovno serijo podatkov in funkcijo forecast sem naredila model, ki napoveduje da bo BDP v prihodnosti še malo padal.\\

\newpage
\begin{center}
\includegraphics[width=\textwidth]{../slike/napovedBDP.pdf}
\end{center}
Na enak način sem naredila model, ki napoveduje
rast prebivalstva v prihodnje. \\
Modela prikazujeta z odtenkom sive tudi interval, v katerem naj bi prikazana spremenljivka ležala v določenem obdobju z določeno verjetnostjo. S temnejšim odtenkom je prikazan interval z 80\% verjetnostjo, s svetlejšim pa 95\% verjetnostjo. Iz grafov je razvidno, da je negotovost glede BDP veliko večja kot za prebivalstvo.

\newpage
\begin{center}
\includegraphics[width=\textwidth]{../slike/napovedPREBIVALSTVA.pdf}
\end{center} 

Čeprav povezave med rastjo BDP in prebivalstva ni pa spodnji graf prikazuje očitno povezanost med deležem BDP in deležem prebivalstva po regijah v Sloveniji v letu 2012. \\
\newpage
\textbf{Legenda za regije na x- osi grafa: Primerjava deleža BDP in prebivalcev po regijah}
\begin{table}[h]
\begin{tabular}{lll}
1 & Obalno kraška \\
2 & Goriška \\
3 & Gorenjska \\
4 & Osrednjeslovenska \\
5 & Notranjsko-kraška \\
6 & Jugovzhodna Slovenija\\
7 & Spodnje posavska\\
8 & Zasavska\\
9 & Savinjska \\
10 & Koroška \\
11 & Podravska\\
12 & Pomurska\\
\end{tabular}
\end{table}

\begin{center}
\includegraphics[width=\textwidth]{../slike/delez.pdf}
\end{center}
Podatke o deležu prebivalstva po regijah sem dobila na spletni strani Statističnega urada:
\begin{itemize}
\item{\url{http://pxweb.stat.si/pxweb/Dialog/varval.asp?ma=05C2001S&ti=&path=../Database/Dem_soc/05_prebivalstvo/10_stevilo_preb/10_05C20_prebivalstvo_stat_regije/&lang=2}} 
\end{itemize}
in jih v obliki csv shranila v tabelo PREBIVALSTVOREGIJE.\\

Kot smo videli v modelu napovedi za BDP je v zadnjih letih BDP začel padati, zato sem naredila stolpični graf, ki prikazuje vpliv padca gospodarske aktivnosti na javne finance.
Za prikaz te posledice sem uvozila podatke v obliki csv iz spletne strani:
\begin{itemize}
\item{\url{http://pxweb.stat.si/pxweb/Dialog/varval.asp?ma=3269302S&ti=&path=../Database/Okolje/32_trajnostni_razvoj/15_medgenerac_sodelovanje/10_32693_drzavni_dolg/&lang=2}}
\end{itemize}
, ki prikazujejo delež javnega dolga v BDP, da sem lahko naredila primerjavo sem poračunala vrednost javnega dolga (tabela DOLG).
Iz grafa je razvidno, da se je dolg Slovenije po padcu BDP začel drastično večati, saj so se odhodki napram prihodkov povečali.

\begin{center}
\includegraphics[width=\textwidth]{../slike/dolg.pdf}
\end{center}


\newpage
\section{Povzetek}
Glavne ugotovitve pri analizi bruto dodane vrednosti v Sloveniji so:
\begin{enumerate}
\item Različne dejavnosti prinašajo različne deleže bruto dodane vrednosti. V Sloveniji največ prispevajo predelovalne dejavnosti, najmanj pa kmetijstvo, gozdarstvo in ribištvo.
\item Skozi čas se delež bruto dodane vrednosti po dejavnostih ne spreminja kaj dosti.
\item Najbolj produktivna regija je Osrednje slovenska, eden izmed razlogov je gotovo bolj gosta naseljenost.
\item Do gospodarske krize (leta 2008) rasteta tako, BDP kot prebivalstvo, po začetku gopodarske krize pa opazimo, da prebivalstvo še naprej raste, BDP pa začne padati.
\item Glavna posledica padca BDP so manjši prihodki države in s tem večji javni dolg.
\end{enumerate}





\end{document}


