\documentclass[11pt,a4paper]{article}

\usepackage[slovene]{babel}
\usepackage[utf8x]{inputenc}
\usepackage{graphicx}


\pagestyle{plain}


\begin{document}
\title{Poročilo pri predmetu \\
Analiza podatkov s programom R}
\author{Maruša Valant}
\maketitle

\section{Analiza regionalne bruto dodane vrednosti v osnovnih cenah po letu,standardni klasifikaciji dejavnosti in regijah} 

V projektu bom analizirala podatke o bruto dodani vrednosti po dejavnostih v Sloveniji v letih 2000-2013. Podatke sem pridobila iz Statističnega urada Republike Slovenije:
-http://pxweb.stat.si/pxweb/Dialog/varval.asp?ma=0309202S&ti=Regionalna+bruto+dodana+vrednost+po+dejavnostih+v+osnovnih+cenah%2C+teko%E8e+cene%2C+Slovenija%2C+letno&path=../Database/Ekonomsko/03_nacionalni_racuni/30_03092_regionalni_rac/&lang=  
-http://www.stat.si/novica_prikazi.aspx?id=6254

\section{Obdelava, uvoz in čiščenje podatkov}

Uvozila sem podatke v obliki csv in spletne strani http://pxweb.stat.si/pxweb/Dialog/varval.asp?ma=0309202S&ti=Regionalna+bruto+dodana+vrednost+po+dejavnostih+v+osnovnih+cenah%2C+teko%E8e+cene%2C+Slovenija%2C+letno&path=../Database/Ekonomsko/03_nacionalni_racuni/30_03092_regionalni_rac/&lang= 
in podatke v obliki html iz spletne strani http://www.stat.si/novica_prikazi.aspx?id=6254.

Po uvozu podatkov sem oblikovala tortni graf, ki predstavlja deleže dodanih vrednosti v Sloveniji po dejavnostih v letu 2012.

\includegraphics{../slike/graf.pdf}

Za primerjavo bruto dodane vrednosti v Sloveniji v letih 2012 in 2013 sem naredila še stolpični graf tako da je za vsako dejavnost skupaj dodana vrednost v letu 2012 (v  modri barvi) in zraven dodana vrednost v letu 2013 (v rdeči barvi). 

Legenda kratic na x-osi v stolpičnem grafu:
A Kmetijstvo in lov,gozdarstvo,ribištvo  
B Rudarstvo	
C Predelovalne dejavnosti	
D Oskrba z el.energijo,plinom in paro	
E Oskr.z vodo;rav.z odpl.,odp.;san.okolja	
F  Gradbeništvo	
G Trgovina;vzdrž.in popravila mot.vozil	
H  Promet in skladiščenje	
I  Gostinstvo	
J Informacijske in komunikacijske dej.	
K Finančne in zavarovalniške dej.	
L Poslovanje z nepremičninami	
M Strokovne,znanstvene in tehnične dej.	
N Druge raznovrstne poslovne dej.	
O Javna uprava in obramba;obv.soc.varnost	
P  Izobraževanje	
Q  Zdravstvo in socialno varstvo	
R Kulturne,razvedrilne in rekreac.dej.	
S  Druge dejavnosti


\includegraphics{../slike/plot.pdf}


\section{Analiza in vizualizacija podatkov}

\includegraphics{../slike/povprecna_druzina.pdf}

\section{Napredna analiza podatkov}

\includegraphics{../slike/naselja.pdf}

\end{document}
