\documentclass[11pt,a4paper]{article}

\usepackage[slovene]{babel}
\usepackage[utf8x]{inputenc}
\usepackage{graphicx}

\pagestyle{plain}

\begin{document}
\title{Poročilo pri predmetu \\
Analiza podatkov s programom R}
\author{Študent FMF}
\maketitle

\section{Analiza regionalne bruto dodane vrednosti v osnovnih cenah po letu,standardni klasifikaciji dejavnosti in regijah}

\section{Obdelava, uvoz in čiščenje podatkov}

\section{Analiza in vizualizacija podatkov}

<<<<<<< HEAD

Za boljšo predstavitev podatkov sem uvozila zemljevid Slovenije. Na njem so označene posamezne regije, glavno mesto Slovenije - Ljuvljana, je označeno z rdečim kvadratom, različne barve regij pa nam povejo kako produktivne so bile posamezne regije leta 2012 ( rumena predstavlja najbolj produktivne, bela pa najmanj produktivne).

\newpage
\begin{center}
 \includegraphics[width=\textwidth]{../slike/Slovenija.pdf}

\end{center}





Iz zemljevida je razvidno, da je leta 2012 največji del bruto dodane vrednosti v Sloveniji prispevala Osrednjeslovenska (36.8\%), najmanj pa Zasavska (1.4\%).

=======
\includegraphics{../slike/povprecna_druzina.pdf}
>>>>>>> db6481ccf92a9100ae48edc8ae14a2c8ee7639a8

\section{Napredna analiza podatkov}

\includegraphics{../slike/naselja.pdf}

\end{document}
